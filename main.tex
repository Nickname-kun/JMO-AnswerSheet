\documentclass{jlreq}
\special{papersize=\the\paperwidth,\the\paperheight}
\usepackage{luatexja}
\usepackage[no-math,deluxe,expert,haranoaji]{luatexja-preset}
\usepackage{fontspec}
\usepackage[bottom=28mm,top=40mm,left=6\zw,right=6\zw]{geometry}
\usepackage{graphicx}
\usepackage{framed}
\usepackage{tikz}
\usetikzlibrary{patterns,calc}
\usepackage[colorlinks=true,linkcolor=black,urlcolor=blue]{hyperref}
\usepackage{xcolor}
\usepackage{mymacro}
\everymath{\displaystyle}
\pagestyle{empty}

\begin{document}
\begin{tabular}{lr}
\begin{minipage}[b]{0.5\linewidth}
\centering
{\large 第34回日本数学オリンピック予選}
\vskip1.5\zw
{\large \textgt{解答用紙}}
\vspace{0.9cm}
\end{minipage}
\begin{minipage}[b]{0.47\linewidth}
\vspace{0pt}
  \centering
  % ----- 右上の受験番号・氏名欄 (調整可能) -----
  \begin{tikzpicture}[x=1mm,y=1mm]
    % パラメータ(必要に応じて変更)
    \def\numDigits{5}      % 受験番号の桁数(小マス数)
    \def\digitW{8}        % 小マスの幅 (mm)
    \def\digitH{11}       % 小マスの高さ (mm)
    \def\nameW{25}        % 氏名欄の幅 (mm)
    \def\nameH{17}        % 氏名欄の高さ (mm)
    \def\gap{0}           % 受験番号群と外枠との左右余白
    \def\outerLW{1.6}     % 外枠の線幅 (pt 相当)
    \def\innerLW{0.6}     % 内部罫線の線幅

    % 全体幅・高さ
    \pgfmathsetmacro{\digitsTotalW}{\numDigits*\digitW}
    \pgfmathsetmacro{\totalW}{\nameW + \gap + \digitsTotalW + \gap}
    \pgfmathsetmacro{\totalH}{\nameH + \digitH} % 上に数字欄、下に氏名欄
    % 外枠
    \draw[line width=\outerLW pt] (0,0) rectangle (\totalW,-\totalH);

    % 上段:受験番号エリア(右寄せの小箱群)
    % 右内側に寄せる計算(外枠の右端から gap をあけて右寄せ)
    \pgfmathsetmacro{\dx}{\totalW - \gap - \digitsTotalW}
    \foreach \i in {0,...,\numexpr\numDigits-1\relax}{
      \pgfmathsetmacro{\x}{\dx + \i*\digitW}
      \draw[line width=\innerLW pt] (\x,0) rectangle ++(\digitW,-\digitH);
    }
    % 「受験番号」ラベル(上段、左寄せ)
    \node[anchor=west,font=\large] at (2.5,-0.5*\digitH) {受験番号};

    % 下段:氏名欄(外枠の左側を占める)
    \draw[line width=\innerLW pt] (0,-\digitH) rectangle (\nameW,-\digitH-\nameH);
    % 氏名ラベル(左上)
    \node[anchor=west,font=\large] at (6,-\nameH-2.5) {氏名};

    % 氏名欄と数字欄の仕切り(垂直線)-- 必要なら太さを変える
    \draw[line width=\outerLW pt] (0,-\digitH) -- (\totalW,-\digitH);
  \end{tikzpicture}
  \vspace{0.3cm}
\end{minipage}
\end{tabular}
\centering
\begin{tikzpicture}[x=1mm,y=1mm]
  % ----- パラメータ(必要なら数値を変えて微調整) -----
  \def\cellW{52}     % 各セル幅(mm)
  \def\cellH{30}     % 各セル高さ(mm)
  \def\headH{8}      % ヘッダ帯高さ(mm)
  \def\rowGap{8}     % 行間(mm)
  \def\leftX{10}     % 左余白(mm)
  \def\topY{10}      % 上余白(mm)
  \def\outerLW{1.6}  % 外枠・仕切り太さ(pt相当の見た目)※描画は数値で指定せずline widthで
  \def\innerLW{0.6}  % ヘッダ線の細さ

  \pgfmathsetmacro{\totalW}{3*\cellW} % 総幅(mm)

  % ----- 各行の外枠と縦仕切り、ヘッダ下の細線を描画 -----
  \foreach \r in {0,1,2,3}{
    % 上端の y 座標(mm単位)
    \pgfmathsetmacro{\y}{\topY + \r*(\cellH + \rowGap)}
    % 外枠(太線)
    \draw[line width=\outerLW pt] (\leftX,-\y) rectangle ++(\totalW,-\cellH);
    % 縦仕切り(太線)
    \foreach \c in {1,2}{
      \pgfmathsetmacro{\x}{\leftX + \c*\cellW}
      \draw[line width=\outerLW pt] (\x,-\y) -- ++(0,-\cellH);
    }
    % ヘッダ下境界(細線): 各セルごと
    \foreach \c in {0,1,2}{
      \pgfmathsetmacro{\x}{\leftX + \c*\cellW}
      \pgfmathsetmacro{\yline}{\y + \headH}
      \draw[line width=\innerLW pt] (\x,-\yline) -- ++(\cellW,0);
    }
  }
  % ----- マクロ:セルに問番号と解答を配置 -----
  \newcommand{\placeCell}[4]{%
    % #1 col(1..3), #2 row(1..4), #3 qno, #4 content
    \pgfmathsetmacro{\px}{\leftX + (#1-1)*\cellW} % 左端 x(mm)
    \pgfmathsetmacro{\py}{\topY + (#2-1)*( \cellH + \rowGap)} % 上端 y(mm)
    % ヘッダ中央(y)
    \pgfmathsetmacro{\hy}{\py + 0.5*\headH}
    % 本文中央(ヘッダ含めた中央)
    \pgfmathsetmacro{\cy}{\py + 0.5*(\headH + \cellH)}
    % 中央 x
    \pgfmathsetmacro{\cx}{\px + 0.5*\cellW}
    % 問番号(ヘッダ内中央)
    \node[font=\Large] at (\cx, -\hy) {#3};
    % 本文(中央)
    \node[inner sep=0pt, align=center] at (\cx, -\cy) {#4};
  }

  % ----- 各セルに内容を配置 -----
  % 1行目
  \placeCell{1}{1}{1}{\huge$\displaystyle\frac{122}{11}$}
  \placeCell{2}{1}{2}{\huge$309,\,311$}
  \placeCell{3}{1}{3}{\huge$5+\sqrt{10}$}

  % 2行目
  \placeCell{1}{2}{4}{\huge$2883$}
  \placeCell{2}{2}{5}{\huge$2519$}
  \placeCell{3}{2}{6}{\huge$\displaystyle\frac{14\sqrt{65}}{13}$}

  % 3行目
  \placeCell{1}{3}{7}{\huge$16\ \mathrm{個}$}
  \placeCell{2}{3}{8}{\huge$2^{990}+1\ \mathrm{個}$}
  \placeCell{3}{3}{9}{\huge$\displaystyle\frac{17}{3}$}

  % 4行目
  \placeCell{1}{4}{10}{\Large${}_{198}\mathrm{C}_{100}\cdot 3\cdot 2^{100}\ \mathrm{通り}$}
  \placeCell{2}{4}{11}{\huge$102050$}
  \placeCell{3}{4}{12}{\huge$\displaystyle\frac{2100^{210}}{2^{164}\cdot 3^{30}}\ \mathrm{個}$}

 \def\smallW{35}      % 小箱の幅(mm)
  \def\smallH{30}      % 小箱の高さ(mm)
  \def\smallVsep{8}    % 第4行との垂直間隔(mm)

  \pgfmathsetmacro{\rightX}{\leftX+\totalW}
  % 小箱の左端 x 座標 = 右端 - 小箱幅(右端揃え)
  \pgfmathsetmacro{\smallLeft}{\rightX - \smallW - 10}
 % 第4行の上端 y(mm):topY + 3*(cellH+rowGap)
  \pgfmathsetmacro{\rowFourY}{\topY + 3*(\cellH + \rowGap)}
  % 第4行の下端 y(mm) = rowFourY + cellH
  \pgfmathsetmacro{\rowFourBottomY}{\rowFourY + \cellH}
  % 小箱の上端 y(mm) = 第4行の下端 + smallVsep
  \pgfmathsetmacro{\smallTopY}{\rowFourBottomY + \smallVsep}
  % 小箱ヘッダ内中央
 \pgfmathsetmacro{\smallcx}{0.5*(\smallLeft + \smallLeft + \smallW)}
 \pgfmathsetmacro{\smallcy}{0.5*(\smallTopY + \smallTopY + 6)}
 
  % 描画(右端揃えで第4行の下に配置)
  \draw[line width=\outerLW pt] (\smallLeft, -\smallTopY) rectangle ++(\smallW, -\smallH);
  \draw[line width=\innerLW pt] (\smallLeft, -\smallTopY-6) -- ++(\smallW,0); 
  \node[font=\small] at (\smallcx, -\smallcy) {合計点};

\end{tikzpicture}
\end{document}
